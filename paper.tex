\documentclass[10pt,a4paper]{article}
\usepackage[english]{babel}
\usepackage{multicol}
\usepackage{multirow}
\usepackage{url,hyperref,graphicx,float,times}
\usepackage{sectsty}
\usepackage{authblk}
\usepackage{textcomp}
\renewcommand{\refname}{}

\setlength{\paperheight}{297mm}
\setlength{\paperwidth}{210mm}
\setlength{\voffset}{-12mm}
\setlength{\topmargin}{0mm}
\setlength{\headsep}{8mm}
\setlength{\headheight}{10mm}
\setlength{\textheight}{235mm}
\setlength{\hoffset}{-4mm}
\setlength{\textwidth}{166mm}
\setlength{\oddsidemargin}{0mm}
\setlength{\evensidemargin}{0mm}
\setlength{\marginparwidth}{0mm}
\setlength{\marginparpush}{0mm}
\setlength{\columnsep}{6mm}
\setlength{\parindent}{6mm}
\setlength{\parskip}{2mm}

%% insert table
%% use as \tabin{size_in_mm}{label}{caption}{table_data}
\newcounter{tabcounter}
\def\tabin #1#2#3#4{
\refstepcounter{tabcounter} \label{#2}
\[ \makebox[#1][c]{#4} \]
%\vspace{0mm}
\begin{center}
  \parbox{7cm}{{\bf Table \arabic{tabcounter}:}\quad {\it #3 } } \\
\end{center}
}

\title{\LARGE
Title
}

\author[*]{\large
{\bf Ching-Chun (Jim) Huang}\thanks{jserv@ccns.ncku.edu.tw}}
\author[**]{\large
{\bf Chung-Fan Yang}\thanks{E24026048@mail.ncku.edu.tw}}
\affil[*]{Department of Computer Science and Information Engineering,
\newline
National Cheng Kung University, Taiwan
\newline
No.1, University Road, Tainan City 701, Taiwan (R.O.C.)}
\affil[**]{Department of Electrical and Electronic Engineering,
\newline
National Cheng Kung University, Taiwan
\newline
No.1, University Road, Tainan City 701, Taiwan (R.O.C.)}
\date{}


\begin{document}

\maketitle

\begin{abstract}
\end{abstract}

\vspace{10mm}

\begin{multicols}{2}

\section{Introduction}

\section{Related Work}

\section{Methodology}

    Based on the works presented above, we had found out several problems in common methodology method used currently.
    First, the methodology tool, e.g. cyclictest, is over-simplified. It only gives the wakeup latency measurements,
    which sometimes is not the main or only performance bottle-neck of a real-time system. Also, general profiling
    tools, e.g. perf, is typically utilizing the sampling methodology method, lacking high fidelity, all-recorded
    profiling result of the target system in microsecond scale, which is usually seen in real-time system tuning. In
    addition, the behavior and loading characteristic of work-load used during test, e.g. hackbench, stress, etc. are
    far from the real target applications. These problems not only would significantly reduce the effectiveness of
    tuning target system, but also will cause actual hazards, harming the real-time performance, obscure.

    Our approach to counter these problems includes a new profiling method and a test work-load, which is able to
    represent the target real-time application. The new profiling tool is derived from paper A Decade of Wasted Cores,
    published in EuroSys 2016. In A decade of Wasted Cores, authors have development a profiler which instruments the
    scheduler of Linux Kernel, extracting scheduling information. Based on this work we have extended the ability of the
    profiler to extract the running entity of each core of the target system also the context-switching information,
    including time and target process id (PID). This enabled us to record and collect kernel scheduling information,
    running PID of each processor core fully during the execution of work-load. The recorded data resolution is based on
    the timer resolution of kernel scheduler. On x86-64 and ARM platform, the resolution could achieve sub-microsecond
    level. This gives us the advantage to justify when would the processor core executes the real-time task, how
    work-loads were scheduled by the kernel scheduler, and random scheduling events. This level of detail could not be
    achieved with the previous mentioned tools, which only utilizes the processor's performance measurement unit (PMU),
    lacking the fidelity requires during the tuning of real-time system.

    We also created a testing framework which resembled target application execution behavior, this application could be
    utilized as the testing work-load during methodology. In addition, application created using this framework has the
    ability to be compiled either as user-space program or kernel module. This would grants us the ability to observe
    the characteristic of target application execution in both user-space and kernel-space. This would gives the insight
    of system-call and user-space I/O overhead. In addition, this framework, while executing, would do a series of
    identically executions and record the execution time of each turn. These data can be used to determined the
    execution jitter of the application on the real-time system, which besides wakeup latency. This could be used to
    reflect to the system-call overhead on the system.

    In aid of these developed tools and work presented above, we proposed a high accuracy methodology system, and apply
    this on various platform, including x86-64 and ARM.
    

\section{Experiment}

\bibliographystyle{unsrt}
\begin{thebibliography}{9}%use this if you have <=9 bib refs
	\bibitem {rtai}{\it https://www.rtai.org/}
\end{thebibliography}

\end{multicols}
\end{document}
